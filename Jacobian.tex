\documentclass{article}
\usepackage[utf8]{inputenc}
\usepackage{amsmath}
\usepackage{environ}
\usepackage{physics}

\NewEnviron{NORMAL}{% 
    \scalebox{5}{$\BODY$} 
} 
 

\title{Using the Jacobian to map directional derivatives between coordinate systems}
\author{Max Sills }
\date{August 2020}

% Defining bank of important matrices for easy reuuse
\newcommand{\showMatrixOne}{\ensuremath{
\begin{bmatrix}
\frac{x}{\sqrt{x^2 + y^2}} & \frac{y}{\sqrt{x^2 + y^2}} \\
\frac{-y}{x^2 + y^2}       & \frac{x}{x^2+y^2}          
\end{bmatrix}
}}

\newcommand{\showMatrixTwo}{\ensuremath{
\begin{bmatrix}
\cos\theta & sin\theta  \\
\frac{-\sin\theta}{r} &  \frac{\cos\theta}{r}
\end{bmatrix}
}}

\newcommand{\showMatrixThree}{\ensuremath{
\begin{bmatrix}
\cos\theta & -r\,\sin\theta  \\
\sin\theta &  r\,\cos\theta
\end{bmatrix}
}}

\newcommand{\showMatrixFour}{\ensuremath{
\begin{bmatrix}
\frac{x}{\sqrt{x^2 + y^2}} & -y \\
\frac{y}{\sqrt{x^2 + y^2}} & x     
\end{bmatrix}
}}
% End matrix bank

\begin{document}
\maketitle

\section{Mapping gradients between coordinate systems}

Start with a function U in cartesian coordinates, and U' for that function expressed in polar coordinates.

We can use the change of coordinates formula to express U' given U


\[ x = r * cos(\theta) \]
\[ y = r * sin (\theta)\]


Now say we were able to express the gradient of U in cartesian coordinates
\[
\nabla U = \frac{\partial U} {\partial x}+
\frac{\partial U} {\partial y}
\]

How could we express the gradient of U' in polar coordinates? Well, we've already been given U' in polar coordinates, so we could compute the gradient directly. However, we can also use the chain rule, the change of coordinates formula, and our knowledge of the gradient in cartesian coordinates to directly derive the gradient in polar coordinates.

$$
\nabla U' = \frac{\partial U'} {\partial r}+
\frac{\partial U'} {\partial \theta}
$$

By the chain rule, 

$$
\frac{\partial U'} {\partial r} = \frac{\partial U'} {\partial x}\frac{\partial x} {\partial r}
+ 
\frac{\partial U'} {\partial y}\frac{\partial y} {\partial r}
$$

$$
\frac{\partial U'} {\partial \theta} = \frac{\partial U'} {\partial x}\frac{\partial x} {\partial \theta}
+ 
\frac{\partial U'} {\partial y}\frac{\partial y} {\partial \theta}
$$

We can rewrite these relations as a covector acting on a Jacobian
\begingroup
\LARGE
\begin{equation}
\left\langle
\begin{matrix}\frac{\partial U}{\partial r} & \frac{\partial U} {\partial \theta}
\end{matrix}
\right\rangle=\left\langle
\begin{matrix}\frac{\partial U}{\partial x} & \frac{\partial U} {\partial y}
\end{matrix}
\right\rangle
*
\begin{bmatrix}
\frac{\partial x}{\partial r} & \frac{\partial y}{\partial r}\\\frac{\partial x}{\partial \theta} & \frac{\partial y}{\partial \theta}
\end{bmatrix}
\end{equation}
\endgroup

We want to find the gradient of U in polar coordinates, which can be represented as the partials of U with respect to r and theta. This is equal to the matrix product of the covector of the gradient in cartesian coordinates which was found above, and the Jacobian, which can be entirely derived from the change of coordinates functions from cartesian to polar.

\section{Examples}

Before we work examples, let's identify the transition functions to and from polar coordinates and the corresponding Jacobians to make things easier.

\subsection{Cartesian to Polar}


\begin{align}
r &= \sqrt{x^2 + y^2} \\
\theta &= \arctan{\left(\frac{y}{x}\right)} 
\end{align}
\begin{equation}
\showMatrixOne
=
\showMatrixTwo
\end{equation}

\subsection{Polar to Cartesian}
\[ x = r * cos(\theta) \]
\[ y = r * sin (\theta)\]

\begin{equation}
\showMatrixThree 
=
\showMatrixFour
\end{equation}

\subsection{Example 1: U = x + y}
\[ U = x + y\]
\[\nabla U = \pdv{U}{x} + \pdv{U}{y} = <1,1>  \]

Now to express U in polar coordinates (r, theta) we find
\[ U = r*cos\theta + r*sin\theta  \]
\[\nabla U = cos\theta + sin \theta \pdv{U}{r} 
+ r (cos\theta - sin \theta \pdv{U}{\theta} \]

\begin{equation}
\begin{bmatrix}
cos + sin & r(cos-sin)
\end{bmatrix}
*
\begin{bmatrix}
\cos\theta & sin\theta  \\
\frac{-\sin\theta}{r} &  \frac{\cos\theta}{r}
\end{bmatrix}
=
\begin{bmatrix}
1 & 1
\end{bmatrix}
\end{equation}

Now, how might we take an arbitrary directional derivative in the first coordinate system and map it to a directional derivative in the second?

Our gradient in the first coordinate system is 
\[\begin{bmatrix}
1 & 1
\end{bmatrix}
\]

When we apply an arbitrary directional derivative (a, b), we produce a scalar:
\begin{displaymath}
\begin{bmatrix}
1 & 1
\end{bmatrix}
*
\begin{bmatrix}
a \\
b
\end{bmatrix}
=
a + b
\end{displaymath}

Applying the Jacobian to the vector this time, instead of the gradient (covector) to the Jacobian, we find:
\begin{displaymath}
\showMatrixTwo
*
\begin{bmatrix}
a \\
b
\end{bmatrix}
=
\begin{bmatrix}
a cos(\theta) + b sin(\theta) \\
\frac{-a sin(\theta)}{r} +
\frac{b cos (\theta)}{r}
\end{bmatrix}
\end{displaymath}

Finally, applying this transformed vector to the gradient we found in polar coordinates we see:


\begin{displaymath}
\begin{bmatrix}
cos + sin & r(cos-sin)
\end{bmatrix}
*
\begin{bmatrix}
a cos(\theta) + b sin (\theta) \\
\frac{-a sin(\theta)}{r} +
\frac{b cos (\theta)}{r}
\end{bmatrix}
=
\end{displaymath}
\begin{displaymath}
\begin{aligned}
(acos + bsin)(cos + sin) + (cos- sin)(bcos -asin) \\
= acos^2 + asin cos + bsin cos + bsin^2 \\
+ bcos^2 - asin cos - bsin cos + a sin^2 \\
= acos^2 + b sin^2 + bcos^2 + a sin^2 \\
= a(cos^2 + sin^2) + b(cos^2 + sin^2) \\
= a + b
\end{aligned}
\end{displaymath}


\subsection{Example 2: U(r, theta) = r *theta}

\[U = r * \theta\]
\[\nabla U = \theta \pdv{U}{r}+ r \pdv{U}{\theta} \]

U(r, theta) in cartesian coordinates:

\[ U(x, y) =\sqrt{x^2 +y^2}
*
\arctan\frac{y}{x} \]

\begin{equation}
\nabla U =
\frac{x \atan(y/x) - y}{\sqrt{x^2 + y^2}}
\pdv{U}{x} 
+
\frac{y \atan(y/x) + x}{\sqrt{x^2 + y^2}}
\pdv{U}{y} 
\end{equation}
\end{document}
